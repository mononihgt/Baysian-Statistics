\documentclass[11pt]{article}
\usepackage{xeCJK} % 加载 xeCJK 宏包以支持中文
\usepackage{fontspec} % 加载 fontspec 宏包以设置字体
\usepackage{indentfirst} % 加载 indentfirst 宏包以设置首行缩进
\usepackage{booktabs}
\usepackage[margin=1in]{geometry} % For setting margins
\usepackage{amsmath} % For Math
\usepackage{cleveref} % For cross-referencing
\crefname{figure}{图}{图} % Define custom names for figures
\Crefname{figure}{图}{图}
\usepackage{fancyhdr} % For fancy header/footer
\setlength{\headheight}{13.6pt} % Adjust head height to avoid warning
\usepackage{graphicx} % For including figure/image
\usepackage{cancel} % To use the slash to cancel out stuff in work
\usepackage{enumitem}
\usepackage{setspace}
\usepackage{diagbox}


\doublespacing
% 设置中文字体和英文字体
\setCJKmainfont{Songti SC Regular} % 设置中文主字体为 STSong,字间距增加
\setmainfont{Times New Roman} % 设置英文字体
\renewcommand{\CJKglue}{\hskip 0.8pt plus 0.08\baselineskip}
\renewcommand{\today}{\number\year 年 \number\month 月 \number\day 日}


%%%%%%%%%%%%%%%%%%%%%%
% Set up fancy header/footer
\pagestyle{fancy}
\fancyhead[LO,L]{2025年春学期}
\fancyhead[CO,C]{贝叶斯统计学基础}
\fancyhead[RO,R]{\today}
\fancyfoot[LO,L]{}
\fancyfoot[CO,C]{\thepage}
\fancyfoot[RO,R]{}
\renewcommand{\headrulewidth}{0.4pt}
\renewcommand{\footrulewidth}{0.4pt}
% 设置首行缩进为一个字符宽度
\setlength{\parindent}{2em} % 1em 约等于一个字符宽度
%%%%%%%%%%%%%%%%%%%%%%

% \title{贝叶斯统计学基础作业1}
% \author{毛沛炫 \ \ \ 3220102692}

\bibliographystyle{plain} % Add bibliography style
\begin{document}
\begin{titlepage}
    \centering
    \vspace*{4cm} % 调整标题的垂直位置
    \Huge
    贝叶斯统计学基础作业2 \\
    \vspace{1cm}
    \Large
    毛沛炫\ \ \ 3220102692 \\
    \vspace{12.3cm}
    \Large
    \today
    \vfill
\end{titlepage}


\noindent 1. 假定对于二项分布参数\textit{p}我们采用均匀先验分布,并且在 10次试验中观察到了 4 次正性结果,
\begin{enumerate}[itemsep=2pt,topsep=0pt,parsep=0pt,label=(\alph*)]
\item  给出先验贝塔分布的参数值(2 分)

\item  给出后验贝塔分布的参数值(2 分)
    
\item  给出在先验分布下二项分布参数 p 的期望值(2 分)
    
\item  给出样本中正性结果的比例(2 分)
    
\item  给出二项分布参数 p 的极大似然估计值(2 分)
    
\item  给出在后验分布下二项参数 p 的期望值,并以先验分布下该参数的期望值和该参数的极大似然估计值的加权平均形式表达(4 分)
\end{enumerate}


\noindent \textbf{解答:}
\begin{enumerate}[itemsep=2pt,topsep=0pt,parsep=0pt,label=(\alph*)]
    \item 先验贝塔分布的参数值
    
    a  = 1, b = 1
    \item 后验贝塔分布的参数值

    a = 5, b = 7

    \item 给出二项分布参数 p 的极大似然估计值
    
    \begin{align}
        f(x) & = p^x(1-p)^{1-x} \\
        L(x_1, x_2, \cdots, x_n; ) & = \prod_{i=1}^{n} p^{x_i}(1-p)^{1-x_i} \\
    \end{align}

    \item 给出样本中正性结果的比例
    
    \item  给出二项分布参数 p 的极大似然估计值
    
    \item  给出在后验分布下二项参数 p 的期望值,并以先验分布下该参数的期望值和该参数的极大似然估计值的加权平均形式表达

\end{enumerate}

\end{document}
